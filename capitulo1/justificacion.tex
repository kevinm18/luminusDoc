\section{Justificación}

El principal motivo de desarrollo de la presente propuesta es el de facilitar a las empresas u organizaciones la implementación de
un ambiente de Big Data, dado que se tiene la creencia de que es un proceso complicado, tardado y muy costoso. Cada uno de los
factores mencionados anteriormente puede ser justificado de la siguiente manera: la complejidad de la implementación de Big Data
se puede reducir mediante la preparación previa del software necesario para la infraestructura que tenga actualmente la empresa; de
igual forma se pueden reducir los tiempos de implementación mediante la automatización de las configuraciones y pre
procesamiento de datos; y finalmente los costos estarán de acuerdo a las computadoras disponibles (al menos 4) que puedan actuar
como nodos de procesamiento y almacenamiento de datos que serán procesados.
\\
Esta herramienta estaría dedicada a empresas u organizaciones que busquen implementar un ambiente de Big Data de una manera
más rápida, sencilla, eficiente y a un menor costo. Para que con su adecuado uso, cualquiera de estas entidades sea capaz de
mejorar, ser más competitiva, y tener un mayor control sobre su propia información, ya que el conocimiento adquirido a partir de
la aplicación de técnicas de minería de datos sobre grandes volúmenes de información estaría siendo de vital importancia para que
dichas organizaciones planeen y ejecuten estrategias que les permitan alcanzar sus objetivos.
\\
Se ha hecho especial hincapié en el sector empresarial en esta sección, ya que, en general, podría decirse que pequeñas y medianas
empresas son los principales clientes potenciales. Esto debido a que la mayoría cuenta con cierta infraestructura de datos y con
bitácoras que contienen información sobre sus ingresos y egresos. Toda esa información será la materia prima que será utilizada
por la herramienta para la generación de conocimiento.
\\
Si bien, existen algunas implementaciones comerciales (las cuales ya se han mencionado en los apartados previos) que realizan la
automatización de la implementación de Big Data, son realmente costosas en recursos de cómputo, y el licenciamiento tiene un
costo elevado con relación a los usuarios y/o equipos a usar. Algunas implementaciones están basadas en máquinas virtuales,
que, si bien, automatizan todo el trabajo, de igual forma resultan costosas con relación al equipo necesario y dinero invertido.
Además, el costo de las herramientas comerciales más robustas puede llegar a los cientos de miles de pesos.
\\
Para la realización de esta propuesta se requerirá dominio en el manejo de bases de datos relacionales y no relacionales;
conocimiento de técnicas de minería de datos y su adecuada aplicación; manejo de herramientas y Frameworks dedicados al
desarrollo de aplicaciones que se ejecuten de manera distribuida; conocimiento en el ámbito de complejidades algorítmicas
para así poder justificar la mejora en la eficiencia de esta herramienta con respecto a las otras ya existentes; manejo de distintos
lenguajes de programación, debido a que el lenguaje que se utilizara para el desarrollo de la aplicación principal será distinto al
lenguaje que se utilizara para la implementación de los algoritmos de minería de datos.
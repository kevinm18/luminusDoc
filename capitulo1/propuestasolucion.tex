\section{Propuesta de solución}
En este trabajo terminal se propone desarrollar un ambiente de Big Data que pueda adaptarse a diferentes casos de estudio o dicho en otras palabras, grupos de datos diferentes donde se desea aplicar un determinado algoritmo, incluyendo un archivo de configuración que permita realizar las adaptaciones correspondientes para cada uno de estos casos de estudio.
\\
El ambiente de Big Data incluirá la aplicación de los siguientes algoritmos de minería de datos: \\
\begin{itemize}
	\item Un algoritmo de reglas de asociación \\
	\item Un algoritmo de arboles de decisión \\
\end{itemize}
Así como la generación de conocimiento y resultados a partir de la aplicación de estos algoritmos.
\\
Haciendo un análisis de los datos correspondientes al caso de estudio en cuestión se determinará si estos pueden soportar y ser aplicados a el algoritmo de minería de datos que el usuario experto desee aplicar. En caso de que esto no sea posible se notificará de esto al usuario experto para este pueda seleccionar otro algoritmo.
\\
Esta solución podrá implementarse en computadoras tradicionales, reduciendo así el costo económico de su implementación en gran medida, ya que se estaría eliminando la necesidad de comprar equipos especializados y ambientarlos.
\\
Debido a que el ambiente de Big Data del que hablamos requiere hacer uso, de varias computadoras tradicionales y cada una de ellas requiere llevar a cabo un proceso de instalación y configuración para que este pueda funcionar. Se desarrollará un instalador que permita simplificar esta tarea realizando todo este proceso desde un solo nodo que será definido como nodo maestro.  
\\
La ejecución y puesta en marcha de los algoritmos de minería de datos se hará desde una Interfaz de Programación de Aplicaciones que podrá ser soportada e incorporada en sistemas que corran haciendo uso del lenguaje de programación JAVA. 
\\
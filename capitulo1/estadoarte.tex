\hypertarget{seccion:IniciarSesion}{\vspace{1pt}}
\newpage
\section{Estado del arte}
Actualmente en el mercado existen diferentes alternativas para los usuarios que busquen hacer uso de Big Data. Las alternativas encontradas, podrían dividirse en 2 tipos:\\
\begin{itemize}
	\item Software que ya contiene algoritmos algoritmos de minería de datos implementados.\\
\begin{table}[H]
	\begin{tabular}{l|l|l}
		SOFTWARE & CARACTERÍSTICAS & PRECIO EN EL MERCADO\\
					\hline                                              
		Knime            & \begin{tabular}[c]{@{}l@{}}-Diferentes maneras de presentar los datos como creación\\ de modelos estadísticos y de minería de datos, como\\ árboles de decisión, máquinas de vector soporte,\\ regresiones, etc.\\ -Aplicación de modelos creados con Knime sobre\\ conjuntos nuevos de datos.\\ -Creación de informes\\ -Posible incorporación de nuevos módulos en lenguaje R o\\ Python por el usuario final.\\ Tamaño de instalación: 1.98GB para la versión gratuita \cite{refi3} \end{tabular}                                                                                                       & \begin{tabular}[c]{@{}l@{}}Gratuita (Menos funcionalidad)/\\ De costo\\ Desde \$44,200 pesos al año por\\ un usuario\\ Hasta \$466,410 pesos al año\\ por 5 usuarios\end{tabular}     
		                        \\
		\hline		                       
		SPSS Modeler IBM & \begin{tabular}[c]{@{}l@{}}-Posible incorporación de nuevos módulos en lenguaje R\\ Python, Hadoop, Spark u otras tecnologías de código\\ abierto por el usuario final.\\ -Diferentes maneras de presentar los datos mediante la\\ creación de modelos.\\ -Acceso y Exportación de datos sin límite de tamaño.\\ -  Más de 30 algoritmos de minería de datos implementados\\ -Ejecuta flujos de datos directamente hacia Spark o\\ Hadoop\\ -Soporte técnico\\ Tamaño de instalación: 1.4GB para la versión de prueba \cite{refi4} \end{tabular}                                                           & \begin{tabular}[c]{@{}l@{}}Gratuita (Prueba 30 días)/De\\ costo\\ Es necesario contactar\\ directamente con IBM para\\ obtener el precio de la\\ herramienta personalizado para\\ cada empresa\end{tabular} \\
		\hline
		Rapid Miner      & \begin{tabular}[c]{@{}l@{}}-Cuenta con gráficos para visualización de información\\ -Comparte reutiliza e implementa modelos predictivos\\ -Ejecuta flujos de datos directamente hacia Hadoop\\ -Presentación de datos desde diferentes representaciones\\ gráficas en tiempo real incluso la creación de árboles de\\ decisión \cite{refi5} \end{tabular} & \begin{tabular}[c]{@{}l@{}}Desde \$2500.00 pesos al año\\ hasta \$10000.00 pesos al año\\ \end{tabular}

\end{tabular} 		                                                         

\end{table}
\item Software de administración de clústers.\\
	\begin{table}[H]
		\begin{tabular}{l|L{9cm}|l}
		SOFTWARE & CARACTERÍSTICAS & PRECIO EN EL MERCADO\\
		\hline
			Databricks & \begin{tabular}[c]{@{}l@{}}-Optimizado para lenguaje R.\\ -Muy simple de operar.\\ -Se enfoca en la usabilidad y en procesos intuitivos de\\ navegación. \cite{infoDatabricks}\\ \end{tabular} & \begin{tabular}[c]{@{}l@{}}-Basic - \$0.07 USD por hora de\\ uso de unidad de procesamiento.\\ -Analytics - \$0.2 USD por hora\\ de uso de unidad de procesamiento.\cite{precioDatabricks} \end{tabular}\\
		\hline
			Cloudera Manager & \begin{tabular}[c]{@{}l@{}}-Plataforma de administración de clústers.\\ -Tiene su propia implementación de búsqueda y ambiente\\ Hadoop con integración y API estándar (CDH).\\ -Cuenta con una interfaz gráfica vía web que opera el stack \\completo de tecnologías de Big Data. \cite{infoCloudera}\\ \end{tabular}  & \begin{tabular}[c]{@{}l@{}}-Gratuita (Prueba de 60 días)\\ -Essentials \$0.06 USD por hora\\ de uso de unidad de procesamiento\\ -Enterprise Data Hub \$0.87 USD \\por hora de uso de unidad de\\ procesamiento. \cite{precioCloudera} \end{tabular} \\
		\hline
			HortonWorks& \begin{tabular}[c]{@{}l@{}}-Completamente open-source.\\ -Cuenta con un set completo de herramientas necesarias \\para un ambiente Hadoop.\\ -Se puede lanzar en la nube o en una máquina virtual.\\ -Gran extensibilidad debido a su naturaleza de código abierto.\\ -Utiliza Ambari como interfaz de usuario. \cite{infoHortonworks}\\ \end{tabular} & \begin{tabular}[c]{@{}l@{}} Gratuito \cite{precioHortonworks}\\ \end{tabular}

		\end{tabular}
	\end{table}
\end{itemize}

Podría decirse que \emph{Luminus} es un híbrido entre las dos clasificaciones anteriormente citadas. Las principales características de \emph{Luminus} son las siguientes:
\begin{itemize}
	\item Trabaja con datos que se encuentra en distintos equipos de cómputo (Red distribuida).\\ 
	\item El usuario puede seleccionar el algoritmo de minería de datos a utilizar para su funcionamiento dentro de un listado de algoritmos disponibles.\\ 
	\item Es capaz de detectar cuando alguno de los nodos de datos es modificado antes de resolver un determinado problema.\\
	\item Los datos son extraídos directamente de los repositorios en tiempo real.\\
	\item Se trabaja con archivos JSON de los datos de entrada.\\ 
\end{itemize}

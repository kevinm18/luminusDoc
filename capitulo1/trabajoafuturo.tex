\section{Trabajo a futuro}
\subsection{Prototipo 3: Página web y algoritmos de minería de datos}
\label{proto3}
Se pretende desarrollar este prototipo en Trabajo Terminal 2 el cual incluirá las siguientes tareas:
\begin{itemize}
	\item Hacer uso de un algoritmo de los tipos de algoritmos de minería de datos que se enlistan a continuación
	\begin{itemize}
		\item Reglas de asociación
		\item Arboles de decisión
	\end{itemize}
	 escrito en java que se adapte al caso de estudio así como su personalización para que pueda ser adaptado a otros casos de estudio de manera sencilla.
	\item Elaboración de las pantallas que conformarán el sitio web, diseño, estructura y contenido que aun no estarán integradas a Luminus por lo tanto, no serán funcionales.  
\end{itemize}
\subsection{Prototipo 4: Algoritmos de minería de datos desde una pagina web}
Se pretende desarrollar este prototipo en Trabajo Terminal 2 el cual comprende el desarrollo de la siguiente tarea:
\\ 
\\ 
Integrar los 2 puntos comprendidos en el prototipo 3 los cuales pueden ser consultados en la sección anterior \ref{proto3} de tal forma que los algoritmos de minera de datos puedan ser accedidos y utilizados desde la pagina web elaborada. 
\\
Además de que estos algoritmos se ejecutarán en el cluster generado en los prototipos Prototipo 1 \nameref{cap:Cap3} y Prototipo 2 \nameref{cap:Cap4} desarrollados en trabajo terminal 1.
\subsection{Prototipo 5: Luminus}
Este prototipo se encontraba diseñado unicamente para convertir este proyecto a un ambiente de análisis de datos adaptable a otros casos de estudio, sin embargo, se vio la necesidad de crear un instalador que permita cargar todo el contenido del Trabajo Terminal al equipo del cliente para que este pueda utilizarlo.\\
Esto con el objetivo de que, el instalador cargue a su equipo todo lo que Luminus necesita para funcionar. y unicamente trabaje sobre el la plantilla de ajuste. \\
Por tal motivo se explicarán los alcances de la plantilla y del instalador por separado.
\subsubsection{Plantilla de ajuste}
Para la parte de la plantilla de ajusteo se busca que se diseñe una plantilla de ajuste la cual permita la adaptación para otros casos de estudio debido a que los algoritmos de Big Data suelen tener diferentes entradas, las cuales cambian de acuerdo para cada caso de estudio.
\\
\subsubsection{Instalador}
Un instalador suele ser un sistema bastante complejo, ya que realiza las siguientes tareas:
\begin{itemize}
	\item Instalación de software
	\item Ejecuta comandos en el shell de la computadora
	\item Verifica software que ya está instalado en la computadora y generalmente es capaz de cambiar la versión
\end{itemize}
Se tiene una parte del instalador desarrollada en Trabajo Terminal 1 la cual carece de robustez, por lo que se pretende que durante el desarrollo del Trabajo Terminal 2, este se vuelva más robusto de lo que es ahora, es decir:
\begin{itemize}
	\item Que sea capaz de reanudar el proceso de instalación cuando éste haya sido interrumpido. Ya sea por decisión del usuario o por un error en el proceso de instalación. El instalador, debe retomar el proceso a partir del punto en el que fue interrumpido.
	\item Que pueda agregar nodos a la red y remover nodos de la red.
	\item Que sea capaz actualizar las IPs de los nodos cuando exista algún cambio en ellas.
	\item Que pueda identificar si alguna de las tecnologías requeridas para la instalación de Luminus, ya existe en alguno de los nodos. De ser así, se validará la versión de la tecnología en cuestión. Si la versión es diferente a la que necesita Luminus, se pregunta al usuario si desea conservar su versión del software, o la versión necesaria para Luminus. Si la versión del software previamente instalada es la correcta, se notifica que Luminus no instaló esa tecnología.
\end{itemize}
Por otro lado, se agregarán al instalador todos los prototipos contenidos en el trabajo terminal.
\subsection{Propuesta de trabajo a futuro para contribuir a esta idea}
En este trabajo a pesar de que se intenta cubrir la mayor cantidad de puntos posibles quedarían abiertos otros puntos a desarrollar. Algunos de los puntos que se sugieren para trabajar posteriormente se listan a continuación:
\begin{itemize}
	\item Hacer un estudio de la cantidad de nodos mas apropiada para un cluster distribuido, antes de que este deje de ser efectivo para entonces recomendar no exceder este numero de equipos.
	\\
	\item Llevar a cabo un análisis completo de cuales son las características mas apropiadas y que impliquen el menor costo posible en la adquisición y mantenimiento de las computadoras que trabajaran en la red distribuida.
	\\
	Con ello ofrecer una propuesta a las empresas, para que consigan equipos de computo con estas características y con ello obtener el mejor rendimiento optimizando su inversión
	\item Hacer una implementación diferente del instalador de \emph{luminus} haciendo uso incluso de otra tecnología que permita hacer validaciones de errores de forma completa y contemplando una mayor cantidad de casos para que este pueda ser mas efectivo.
	\item Realizar el desarrollo y adaptación a Luminus de otros algoritmos de minería de datos ya sea de arboles de decisión o de reglas de asociación o bien crear un nuevo conjunto de algoritmos. 
	\\
	Para que el análisis que se realice a los datos pueda llegar a ser mas completo y existan mas opciones para el usuario experto de algoritmos a aplicar a su caso de estudio.
	\item Realizar una implementación que permita obtener automaticamente datos de una fuente externa y los actualice en tiempo real dentro del repositorio de datos. Esto para empresas que poseen datos cambiantes y requieren actualización constante de los mismos.    
\end{itemize}   
Cabe señalar que además de estas propuestas existen muchas otras que pueden adaptarse a este proyecto y ampliarlo para diferentes finalidades. Las posibilidades son prácticamente infinitas y dependen directamente de las necesidades especificas de cada usuario o empresa. 
\\
Las propuestas que aquí se señalan son ideas que se encontraron durante el desarrollo del proyecto y que se considera pueden contribuir en gran medida al mismo tomando en cuenta los objetivos que se establecieron.

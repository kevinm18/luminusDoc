\section{Anexo B: Código del instalador}
Código de la clase principal luminus-ins.sh
\begin{lstlisting} 
#!/bin/bash
# Instalador general de luminus version 4.

# El instalador pide al usuario introducir las ips de los nodos incluida la del master.
# Las ips son almacenadas en un archivo de texto.

# Funcion que valida el formato de las ips mediante el uso de expresiones regulares.
function validar_ip() {
	local valida=0
	local ip=$1
	if [[ $ip =~ ^[0-9]{1,3}\.[0-9]{1,3}\.[0-9]{1,3}\.[0-9]{1,3}$ ]]; then
	    OIFS=$IFS
	    IFS='.'
	    ip=($ip)
	    IFS=$OIFS
	    if [[ ${ip[0]} -le 255 && ${ip[1]} -le 255 \
	        && ${ip[2]} -le 255 && ${ip[3]} -le 255 ]]
	    then
	    	valida=1
	    	echo $valida
	    else
	    	echo $valida
	    fi
	else
		echo $valida
	fi
}

# Lectura de la ip del nodo maestro
function pedir_ip_maestro() {
	# Se escribe un 1 en el log del progreso de instalacion.
	# El 1 corresponde a que solmente se inicio pero no se completo el primer paso.
	echo "1" > luminus-ins.log
	cp /etc/hosts.back /etc/hosts
	confirmacionIPMaestro=0
	while [ $confirmacionIPMaestro -eq 0 ]
	do
		echo "Introduce la ip del nodo maestro:"
		read ipMaestro
		if [ $(validar_ip $ipMaestro) -eq 1 ]
		then
			echo "Introduce el nombre del nodo maestro:"
			read nodoMaestro
			echo $nodoMaestro > nombres.txt
			echo $ipMaestro > ips.txt
			echo "$ipMaestro $nodoMaestro maestro" >> /etc/hosts
			echo "$ipMaestro $nodoMaestro maestro" > hosts.maestro
			cp /etc/hosts /etc/hosts.back
			cp ~/.bashrc ~/bashrc.back
			ssh-keygen -f "/root/.ssh/known_hosts" -R $ipMaestro
			confirmacionIPMaestro=1
		else	
			echo "Introduzca una IP valida."
		fi
	done
	# Se escribe un 2 en el log de progreso de instalacion.
	# El 2 corresponde a que se intrudujo de manera satisfactoria la IP del nodo 
	maestro.
	echo "2" > luminus-ins.log
	pedir_ips_esclavos
}

# Lectura de las ips de los nodos esclavos. 
# Escritura de las ips en el archivo hosts del nodo maestro.
# Adicion de la ip del nodo maestro al archivo hosts de los nodos esclavos.
function pedir_ips_esclavos() {
	confirmacion=1
	confirmacionIP=0
	contador=1
	while [ $confirmacion -eq 1 ]
	do
		while [ $confirmacionIP -eq 0 ]
		do
			echo "Introduce la ip del nodo $contador:"
			read ipNodo
			if [ $(validar_ip $ipNodo) -eq 1 ]
			then
				ping -c1 -qq $ipNodo
				if [ $? -ne 0 ]
				then
					echo "Host no encontrado."
				else
					echo "Host encontrado!"
					echo "Introduzca el nombre del host:"
					read nodo
					echo "$ipNodo $nodo $nodo" >> /etc/hosts
					ssh-copy-id -i ~/.ssh/id_rsa.pub root@$ipNodo
					ssh root@$ipNodo "cp /etc/hosts.back /etc/hosts"
					scp hosts.maestro root@$ipNodo:/etc/
					ssh root@$ipNodo "cat /etc/hosts.maestro >> 
					/etc/hosts"
					ssh root@$ipNodo "rm /etc/hosts.maestro"
					ssh root@$ipNodo 
					"cp /etc/hosts /etc/hosts.back"
					ssh root@$ipNodo "cp ~/bashrc
					 ~/bashrc.back"
					ssh-keygen -f "/root/.ssh/known_hosts" -R 
					$ipNodo
					echo $nodo >> nombres.txt
					echo $ipNodo >> ips.txt
					contador=$((contador + 1))
					confirmacionIP=1
				fi
			else
				echo "Introduzca una IP valida."
			fi
		done
		echo "Si desea agregar otro nodo escriba 1. Si no, escriba 0. (1/0)"
		read confirmacion
	done
	# Numero de replicacion para el hdfs segun el numero de nodos esclavos
	cp config-files/hdfs-site-config.back.xml config-files/hdfs-site-config.xml
	numNodos=$((contador - 1))
	echo $numNodos >> config-files/hdfs-site-config.xml
	echo "</value> </property> </configuration>" >> 
	config-files/hdfs-site-config.xml
	# Se escribe un 3 en el log de progreso de instalacion.
	# El 3 corresponde a que se introdujeron de manera satisfactoria las IPs 
	de los esclavos.
	echo "3" > luminus-ins.log
	instalar_java_maestro
}

# Instalacion de java 1.8 especifico para hadoop
function instalar_java_maestro() {
	rm -rf /usr/lib/jvm/java-8-oracle-hadoop
	mkdir /usr/lib/jvm
	tar -xvf jdk-8u201-linux-x64.tar.gz
	mv jdk1.8.0_201/ /usr/lib/jvm/java-8-oracle-hadoop
	update-alternatives --install /usr/bin/java java 
	/usr/lib/jvm/java-8-oracle-hadoop/bin/java 1061
	# Se escribe un 4 en el log de progreso de instalacion.
	# El 4 corresponde a la instalacion de java de manera satisfactoria en el
	 nodo maestro.
	echo "4" > luminus-ins.log
	instalar_java_esclavos
}

function instalar_java_esclavos() {
	contador=1
	while read line
	do
		if [ $contador -ne 1 ]
		then
			scp jdk-8u201-linux-x64.tar.gz root@$line:/home/
			ssh -n root@$line "rm -rf /usr/lib/jvm/java-8-oracle-hadoop"
			ssh -n root@$line "mkdir /usr/lib/jvm/"
			ssh -n root@$line "tar -xvf /home/jdk-8u201-linux-x64.tar.gz"
			ssh -n root@$line "mv jdk1.8.0_201/ 
			/usr/lib/jvm/java-8-oracle-hadoop"
			ssh -n root@$line "update-alternatives --install /usr/bin/java java 
			/usr/lib/jvm/java-8-oracle-hadoop/bin/java 1061"
		fi
		contador="$((contador + 1))"
	done < ips.txt
	cp  ~/.bashrc.backup ~/.bashrc

	# Se escribe un 5 en el log de progreso de instalacion.
	# El 5 corresponde a la instalacion de java de manera satisfactoria en los 
	esclavos.
	echo "5" > luminus-ins.log
	#instalar_scala
}

#Instalacion de Scala en todos los nodos.
function instalar_scala() {
	./scala-ins.sh
	contador=1
	while read line
	do
		if [ $contador -ne 1 ]
		then
			echo $line
			ssh -qq root@$line < scala-ins.sh
		fi
		contador="$((contador + 1))"
	done < ips.txt

	# Se escribe un 6 en el log de progreso de instalacion.
	# El 6 corresponde a la instalacion de scala de manera satisfactoria en el 
	maestro y los esclavos.
	echo "6" > luminus-ins.log
	instalar_spark
}

# Instalacion de spark en todos los nodos.
function instalar_spark() {
	./spark-ins.sh
	contador=1
	while read line
	do
		if [ $contador -ne 1 ]
		then
			echo $line
			ssh -qq root@$line < spark-ins.sh
		fi
		contador="$((contador + 1))"
	done < ips.txt

	# Se escribe un 7 en el log de progreso de instalacion.
	# El 7 corresponde a la instalacion de Spark de manera satisfactoria en el 
	maestro y los esclavos.
	echo "7" > luminus-ins.log
	configurar_spark_maestro
}

# Configuracion de Spark en el nodo maestro.
function configurar_spark_maestro() {
	# Copia de la plantilla de esclavos de Spark.
	cp /opt/spark/conf/slaves.template /opt/spark/conf/slaves

	# Se agregan los esclavos y el maestro a la lista de nodos de la red.
	cat ips.txt > /opt/spark/conf/slaves 

	# Se copia la plantilla de spark-env
	cp /opt/spark/conf/spark-env.sh.template /opt/spark/conf/spark-env.sh

	# Se agrega el maestro al archivo spark-env.sh
	echo "export SPARK_MASTER_HOST=$ipMaestro" >> /opt/spark/conf/spark-env.sh
	
	# Se escribe un 8 en el log de progreso de instalacion.
	# El 8 corresponde a la configuracion de Spark de manera satisfactoria en el
	 maestro.
	echo "8" > luminus-ins.log
	instalar_hadoop_maestro
}

# Instalacion de Hadoop.
function instalar_hadoop_maestro() {
	# wget http://mirrors.sonic.net/apache/hadoop/commo
	n/hadoop-3.1.1/hadoop-3.1.1.tar.gz
	rm -rf /opt/hadoop
	mkdir /opt/hadoop
	tar -xzvf hadoop-3.1.1.tar.gz
	mv hadoop-3.1.1/* /opt/hadoop/

	# Archivo de configuracion de las variables de entorno
	cat config-files/config-bash.txt >> ~/.bashrc
	source ~/.bashrc

	# Modificacion del arhcivo de configuracion hadoop-env.sh
	# Respaldo del archivo hadoop-env.sh
	cp /opt/hadoop/etc/hadoop/hadoop-env.sh 
	/opt/hadoop/etc/hadoop/hadoop-env.backup.sh 
	# Limpieza del archivo
	# echo "" > /opt/hadoop/etc/hadoop/hadoop-env.sh
	# Sustitucion por el archivo con la variable de entorno JAVA_HOME con su valor 
	correcto
	cp config-files/hadoop-env-config.sh /opt/hadoop/etc/hadoop/hadoop-env.sh

	# Modificacion del archivo de configuracion core-site.xml
	# Respaldo del archivo core-site.xml
	cp /opt/hadoop/etc/hadoop/core-site.xml 
	/opt/hadoop/etc/hadoop/core-site.backup.xml
	# Limpieza del archivo 
	# echo "" > /opt/hadoop/etc/hadoop/core-site.xml
	# Sustitucion por el archivo con la configuracion correcta
	cp config-files/core-site-config.xml 
	/opt/hadoop/etc/hadoop/core-site.xml

	# Modificacion del archivo de configuracion hdfs-site.xml
	# Respaldo del archivo hdfs-site.xml
	cp /opt/hadoop/etc/hadoop/hdfs-site.xml 
	/opt/hadoop/etc/hadoop/hdfs-site.backup.xml
	# Limpieza del archivo
	# echo "" > /opt/hadoop/etc/hadoop/hdfs-site.xml

	# Sustitucion por el archivo con la configuracion correspondiente
	cp config-files/hdfs-site-config.xml 
	/opt/hadoop/etc/hadoop/hdfs-site.xml

	# Configuracion de los archivos mapred y yarn.
	./parser.sh

	# Modificacion del archivo de configuracion mapred-site.xml
	# Respaldo del archivo mapred-site.xml
	cp /opt/hadoop/etc/hadoop/mapred-site.xml 
	/opt/hadoop/etc/hadoop/mapred-site.backup.xml
	# Limpieza del archivo
	# echo "" > /opt/hadoop/etc/hadoop/mapred-site.xml
	# Sustitucion por el archivo con la configuracion correspondiente
	cp config-files/mapred-site-config.xml 
	/opt/hadoop/etc/hadoop/mapred-site.xml

	# Modificacion del archivo de configuracion yarn-site.xml
	# Respaldo del archivo yarn-site.xml
	cp /opt/hadoop/etc/hadoop/yarn-site.xml 
	/opt/hadoop/etc/hadoop/yarn-site.backup.xml
	# Limpieza del archivo
	# echo "" > /opt/hadoop/etc/hadoop/yarn-site.xml
	# Sustitucion por el archivo con la configuracion correspondiente
	cp config-files/yarn-site-config.xml /opt/hadoop/etc/hadoop/yarn-site.xml

	# Configuracion del archivo workers
	contador=1
	while read line
	do
		if [ $contador -ne 1 ]
		then
			echo $line >> /opt/hadoop/etc/hadoop/workers
		fi
		contador="$((contador + 1))"
	done < ips.txt

	# Se escribe un 9 en el log de progreso de instalacion.
	# El 9 corresponde a la configuracion de Hadoop de manera satisfactoria 
	#en el maestro.
	echo "9" > luminus-ins.log
	instalar_hadoop_esclavos
}

function instalar_hadoop_esclavos() {
	# Copiado de los archivos de configuracion de hadoop del nodo maestro 
	#a los nodos esclavos
	contador=1
	while read line
	do
		if [ $contador -ne 1 ]
		then
			echo "Instalando Hadoop en nodo $contador ($line)..."
			scp hadoop-3.1.1.tar.gz root@$line:/home/
			ssh -qq root@$line < hadoop-ins.sh
			scp config-files/core-site-config.xml 
			root@$line:/opt/hadoop/etc/hadoop/core-site.xml
			scp config-files/hadoop-env-config.sh 
			root@$line:/opt/hadoop/etc/hadoop/hadoop-env.sh
			scp config-files/hdfs-site-config.xml 
			root@$line:/opt/hadoop/etc/hadoop/hdfs-site.xml
			scp config-files/mapred-site-config.xml 
			root@$line:/opt/hadoop/etc/hadoop/mapred-site.xml
			scp config-files/yarn-site-config.xml 
			root@$line:/opt/hadoop/etc/hadoop/yarn-site.xml
			scp /opt/hadoop/etc/hadoop/workers 
			root@$line:/opt/hadoop/etc/hadoop/workers
			contador=$((contador + 1))
		fi
		contador="$((contador + 1))"
	done < ips.txt

	# Se escribe un 10 en el log de progreso de instalacion.
	# El 10 corresponde a la configuracion e instalacion de Hadoop en los nodos 
	#esclavos 
	# de manera satisfactoria en el maestro.
	echo "10" > luminus-ins.log
}

echo "*************** Bienvenido a Luminus ***************"

pedir_ip_maestro
instalar_hadoop_maestro

# # Ciclo para recolectar la informacion del archivo log
 contador=1
 while read line
 do
 	if [ $contador -eq 1 ] 
 	then
 		line=$line
 		break
 	fi
 done < luminus-ins.log

 echo $line

# # Switch para saber en que paso iniciara el instalador
 if [ $line -eq '1' ] || [ $line -eq '10' ]
 then
# 	pedir_ip_maestro
 fi
 if [ $line -eq '2' ]
 then
 	echo "REANJUNDANDO INSTALACION..."
# 	pedir_ips_esclavos
 fi
 if [ $line -eq '3' ]
 then
 	echo "REANJUNDANDO INSTALACION..."
# 	instalar_java_maestro
 fi
 if [ $line -eq '4' ]
 then
 	echo "REANJUNDANDO INSTALACION..."
# 	instalar_java_esclavos
 fi
 if [ $line -eq '5' ]
 then
 	echo "REANJUNDANDO INSTALACION..."
# 	instalar_scala
 fi
 if [ $line -eq '6' ]
 then
 	echo "REANJUNDANDO INSTALACION..."
 	instalar_spark
 fi
 if [ $line -eq '7' ]
 then
 	echo "REANJUNDANDO INSTALACION..."
 #	configurar_spark_maestro
 fi
 if [ $line -eq '8' ]
 then
 	echo "REANJUNDANDO INSTALACION..."
# 	instalar_hadoop_maestro
 fi
 if [ $line -eq '9' ]
 then
 	echo "REANJUNDANDO INSTALACION..."
 #	instalar_hadoop_esclavos
 fi
\end{lstlisting} 
Código de scala-ins.sh
\begin{lstlisting}
# Instalacion de Scala
wget https://downloads.lightbend.com/scala/2.13.0-M1/scala-2.13.0-M1.deb
dpkg -i scala-2.13.0-M1.deb
scala -version
\end{lstlisting}
Código de spark-ins.sh
\begin{lstlisting}
# Instalacion de Spark
rm spark-2.1.0-bin-hadoop2.7.tgz
wget http://d3kbcqa49mib13.cloudfront.net/spark-2.1.0-bin-hadoop2.7.tgz
mkdir /opt/spark
tar -xzvf spark-2.1.0-bin-hadoop2.7.tgz
cp -a spark-2.1.0-bin-hadoop2.7/. /opt/spark/
rm -rf spark-2.1.0-bin-hadoop2.7
echo 'export SPARK_HOME=/opt/spark/' >> ~/.bashrc
echo 'export PATH="/opt/spark/bin/:/opt/spark/sbin/:$PATH"' >> ~/.bashrc
source ~/.bashrc
\end{lstlisting}
Código de hadoop-ins.sh
\begin{lstlisting}
# Inicia instalacion de hadoop
# Descarga, descompresion y movimiento del
# contenido de hadoop a su carpeta

#wget http://mirrors.sonic.net/apache/hadoop/common/hadoop-3.1.1/hadoop-3.1.1.tar.gz
rm -rf /opt/hadoop
mkdir /opt/hadoop
tar -xzvf /home/hadoop-3.1.1.tar.gz
mv hadoop-3.1.1/* /opt/hadoop/
\end{lstlisting}
Código de backup.sh
\begin{lstlisting}
# Se ejecuta en el nodo maestro.
cp /etc/hosts /etc/hosts.back
cp ~/.bashrc ~/bashrc.back

# Se ejecuta en cada uno de los nodos.
contador=1
while read line
do
	if [ contador -ne 1 ]
		ssh root@$line "cp /etc/hosts /etc/hosts.back"
		ssh root@$line "cp ~/bashrc ~/bashrc.back"
	then
	contador="$((contador + 1))"
	fi 
done
\end{lstlisting}
Código de parser.sh
\begin{lstlisting}
#!/usr/bin/env bash
function parsear_conjunto_valores_hdp() {
rm -rf valores_hdp.txt
while IFS='=' read -ra line; do
  for i in "${line[@]}"; do
      while IFS=' ' read -ra value; do
      	for j in "${value[1]}"; do
      		echo $value >> valores_hdp.txt
      	done
      done <<< $i
  done
done <<< $(echo $(python hdp_conf_files/scripts/yarn-utils.py -c 1 -m 1 -d 1 false))
}
function generar_property() {
	echo "<property>" >> $3
	echo "<name>$1</name>" >> $3
	echo "<value>$(sed $2'q;d' valores_hdp.txt)</value>" >> $3
	echo "</property>" >> $3
}

# Metodo que construye el archivo mapred-site-config.xml.
function generar_mapred_site_xml() {
	cp config-files/mapred-site-config.back.xml config-files/mapred-site-config.xml
	# Property para el valor yarn.app.mapreduce.am.command-opts
	# generar_property 
	#'yarn.app.mapreduce.am.command-opts' '23' 
	#'config-files/mapred-site-config.xml'
	# Property para el valor mapred.map.memory.mb
	generar_property 'mapreduce.map.memory.mb' 
	'18' 'config-files/mapred-site-config.xml'
	# Property para el valor mapreduce.reduce.memory.mb
	generar_property 'mapreduce.reduce.memory.mb' '20' 
	'config-files/mapred-site-config.xml'
	# Property para el valor yarn.app.mapreduce.am.resource.mb
	generar_property 'yarn.app.mapreduce.am.resource.mb' '22' 
	'config-files/mapred-site-config.xml'
	# Property para el valor mapreduce.map.java.opts
	# generar_property 'mapreduce.map.java.opts' '19' 
	#'config-files/mapred-site-config.xml'
	# Property para el valor mapreduce.java.opts
	# generar_property #'mapreduce.reduce.java.opts' '21' 
	#'config-files/mapred-site-config.xml'
	# Property para el valor mapreduce.task.io.sort.mb
	# generar_property 
	#'mapreduce.task.io.sort.mb' '24' 'config-files/mapred-site-config.xml'

	echo '</configuration>' >> config-files/mapred-site-config.xml 
}

# Metodo que construye el archivo yarn-site-config.xml
function generar_yarn_site_xml() {
	cp config-files/yarn-site-config.back.xml config-files/yarn-site-config.xml
	# Property para el valor yarn.nodemanager.resource.memory-mb
	generar_property
	'yarn.nodemanager.resource.memory-mb' '17' 
	'config-files/yarn-site-config.xml'
	# Property para el valor yarn.scheuler.maximum.allocation-mb
	generar_property 'yarn.scheduler.maximum.allocation-mb' '16' 
	'config-files/yarn-site-config.xml'
	# Property para el valor yarn.scheuler.minimum.allocation-mb
	generar_property 'yarn.scheduler.mimimum.allocation-mb' '15' 
	'config-files/yarn-site-config.xml'
	# Property para el valor yarn.app.mapreduce.am.command-opts
	# generar_property 'yarn.app.mapreduce.am.command-opts' '23' 
	'config-files/yarn-site-config.xml'
	echo '</configuration>' >> config-files/yarn-site-config.xml
}

parsear_conjunto_valores_hdp
generar_yarn_site_xml
generar_mapred_site_xml
\end{lstlisting}
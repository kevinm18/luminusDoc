\section{Conclusiones}
Puede decirse con toda certeza que los objetivos que se establecieron al inicio del proyecto se cumplieron, esto indica, que se pudo completar lo que se esperaba realizar durante el desarrollo de este proyecto. 
\\
Se realizaron algunos cambios en cuanto a las tecnologías propuestas para ser utilizadas y los prototipos que se establecieron en un principio, sin embargo, para cada cambio realizado se busco primeramente que este siguiera cumpliendo los objetivos que fueron planteados en un inicio. Con la finalidad de que ningún cambio nos alejará de dichos objetivos.
\\
Sobre la marcha del presente trabajo terminal, se detectó que era necesario realizar un instalador que fuera capaz de simplificar la preparación del ambiente de análisis de datos \emph{Luminus} al usuario experto.  
\\
Esta necesidad fue detectada debido a que el número de pasos necesario para tener un ambiente de análisis de datos que pueda empezar a operar era grande. Por lo que, se consideró simplificar este proceso para el usuario experto con el objetivo de que pudiera centrarse en la utilización del ambiente de minería de datos, y no hacer tan engorroso su proceso de instalación.
\\
El instalador desarrollado cuenta con múltiples validaciones que se hicieron para englobar la mayor cantidad de errores posibles e instala todas las tecnologías necesarias para tener en funcionamiento un ambiente de análisis de datos. 
\\
Por lo que, se puede concluir que el instalador fue una implementación que logro cumplir el objetivo para el que se propuso, además de que se ha comprobado su correcto funcionamiento ya que hasta ahora se han realizado múltiples instalaciones.
\\
Por otro lado, también se interactuo con el ambiente de análisis de datos haciendo las instalaciones correspondientes en equipos de cómputo tradicionales y se aplicaron algoritmos de minería de datos sobre esta instalación con la finalidad de mostrar que con equipos de cómputo tradicionales es posible hacer uso de Big Data y que estos son capaces de soportar estas operaciones. Por lo que se puede decir, que en un momento dado no es absolutamente necesario para una empresa invertir en equipo de computo especializado para el manejo de Big Data.
\\
Los algoritmos de minería de datos implementados se encuentran en funcionamiento y se obtienen los resultados esperados por los mismos, con lo que de acuerdo a las pruebas se pudo comprobar su correcta implementación y la demostración de que en caso de programarse nuevos algoritmos de minería de datos este ambiente serio capaz de soportarlos para su ejecución.
\\
No se pretende limitar a que únicamente opere con los algoritmos implementados sino la intensión de esta implementación es demostrar que para diferentes algoritmos propuestos el ambiente puede realizar los cálculos pertinentes.
\\ 
Esto podría abrir las puertas a que muchas empresas comiencen a utilizar a estas técnicas para el análisis de sus datos sin tener que realizar una gran inversión inicial, por lo que, podría constituir la posibilidad de crecimiento en el uso de Big Data en la industria.
